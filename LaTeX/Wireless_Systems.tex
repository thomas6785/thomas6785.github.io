\documentclass[12pt]{article}


%\setcounter{section}{-1}
% Style guide
% Each 'section' is a chapter (PDF) on Brightspace
% They are 1-indexed in accordance with the notes on BS

\usepackage{enumitem}
\usepackage{amsmath}
\usepackage{parskip}
\usepackage{outlines}
\usepackage{framed}
\usepackage{graphicx}
\usepackage{xcolor}
\pagecolor{black}
\color{white}

\begin{document}

\tableofcontents

\begin{flushright}[Lecture on 1.2]\end{flushright}

\section{Introduction}
Dr. Declan Delaney

\subsection{Motivation and Introduction}
\begin{outline}[enumerate]
\1 Signals travel without wires
    \2 In this module, signals travel as radio waves\\
    (optical and acoustic systems ignored)
\1 Applications are mostly in communications
    \2 Signals modulated to carry information
    \2 Many familiar applications such as radar, navigation, etc.
\end{outline}

\begin{framed}
Example: Modern smart phone has approximately 9 distinct wireless systems. Try identifying them?


\begin{itemize}[noitemsep]
    \item NFC
    \item Cellulars
    \begin{itemize}[noitemsep]
        \item 2G
        \item 3G
        \item 4G
        \item 5G
    \end{itemize}
    \item GPS
    \item Bluetooth
    \item WiFi
    \item UWB
    \item Lidar
\end{itemize}
\end{framed}

\textbf{Advantages of Wireless}

\begin{itemize}[noitemsep]
    \item Mobility
    \item Good for one-to-many transmissions
    \item Cheap
\end{itemize}
Increasingly used for high capacity point-to-point links (cheaper than wired) (e.g. to serve remote areas)

\textbf{Advantages of Wired}

\begin{itemize}[noitemsep]
    \item Very little leakage
    \item No interference
    \item Multiple systems can operate adjacently without issue
\end{itemize}
but considerably more overheads.
Suitable for super high capacity lines (eg. fibre-optic transatlantic cables)

\subsection{The Wireless Spectrum}
The EM spectrum is a shared and limited resource.\\
Mostly regulated by government agencies.

Frequencies must be carefully given out, but can be reused at different locations as we will see.

Overview of a wireless system:
\begin{itemize}[noitemsep]
\item Start with raw data
\item Source coding (compression)
\item Channel coding (error detection \& error correction)
\item Modulation
\item TX
\item RX
\item Demodulation
\item Channel decoding
\item Source decoding
\end{itemize}

This module is mainly about modulation and TX/RX, the rest is information theory.

\subsection{Assessment \& Delivery}
    \begin{tabular}{ l l l }
        \textbf{Component} & \textbf{Timing} & \textbf{Weight} \\
        Lab Assignments & Varied (3 labs) & 25\% \\
        Online BS quizzes & ? & 25\% \\
        Final Exam & & 50\%
    \end{tabular}


\begin{itemize}[noitemsep]
    \item Lab 1: Receiver architectures
    \item Lab 2: Phase-Locked Loops
    \item Lab 3: Amplifiers
\end{itemize}


Open book final with emphasis on design and problem solving

\subsection{Module Outline}

\begin{itemize}[noitemsep]
    \item (1) Radio Link Design
    \begin{itemize}[noitemsep]
        \item Link budget?
        \item How far? How much power?
    \end{itemize}
    \item (2) Non-Linear System
    \item (3) Frequency Generation and Synthesis
    \item (4) Transmitter Design
    \begin{itemize}[noitemsep]
        \item Requirements and specifications
        \item Transmitter architecture choices
    \end{itemize}
    \item (5) Noise
    \begin{itemize}[noitemsep]
        \item Sources of noise
        \item Noise analysis, low-noise design
    \end{itemize}
    \item (6) Receiver Design
    \begin{itemize}[noitemsep]
        \item Requirements and specifications
        \item Receiver architecture choices
    \end{itemize}
    \item (7) Transceiver Design
    \begin{itemize}[noitemsep]
        \item Transmitter and receiver combined!
    \end{itemize}
    \item (8) Antennas and Propagation
    \begin{itemize}[noitemsep]
        \item Review of antenna theory
        \item Practical antennas and propagation of radio waves
    \end{itemize}
    \item (9) System-Level Issues and Examples
\end{itemize}

\subsection{Textbooks}
Purely optional, module notes should be sufficient.

\begin{itemize}[noitemsep]
    \item "Microwave and RF Design of Wireless Systems"

    by David M. Pozar
    \item "Antennas"

    by John D. Kraus
\end{itemize}

\begin{flushright}[Lecture on 1.3]\end{flushright}

\subsection{Basics of Wireless Communication}
Amplifier to increase signal power enough to drive the antenna

\textbf{Multiple Access}
\begin{itemize}[noitemsep]
\item CSMA: Listen to the channel, send if it's clear
\item FDMA: Frequency divided MA
\item TDMA: Time divided MA
\end{itemize}

You require a 'guard band' between frequency bands where no data is sent to avoid interference

CDMA is a good way to overcome this waste, ODMA is an even better approach

\subsubsection{Main components of a transmitter}
\begin{itemize}[noitemsep]
    \item Signal is 'mixed' (modulated) with an oscillator at the frequency of the channel being used
    \begin{itemize}[noitemsep]
        \item The frequency has to be adjustable to allow for different channels
    \end{itemize}
    \item A power amp is required to power an antenna
\end{itemize}
Amp goes first because high-frequency amplification is a fucking nightmare

\subsubsection{Main components of a receiver}
Signal arrives on an antenna (which collects EM waves in the vicinity, sometimes in a preferred direction, and puts them on a cable, waveguide, or circuit board track)

RX:
\begin{itemize}[noitemsep]
    \item Receive at very low power
    \item Select and amplify the desired signal
    \item Estimate the original signal
\end{itemize}

The signal is too weak to demodulate, so you need a high-frequency amplifier before the demodulator.

For most of this module, we will look at block-level circuits and not worry about precise circuit design.

Channel capacity (Shannon-Hartley):
\begin{math}
C = B \times\text{log}_2(1+\frac{S}{N})
\end{math}

\begin{itemize}[noitemsep]
    \item C is channel capacity (eg. bits per second)
    \item B is bandwidth (Hz)
    \item $\frac{S}{N}$ is the signal-to-noise ratio
\end{itemize}

Cellular:
\begin{itemize}[noitemsep]
    \item 2G operated at 800-900 MHz and 64 kHz channels
    \item 3G operated at 1-2 GHz and got 8 MHz channels
    \item 4G operates at up to 5 GHz and 50-100 MHz channels
    \item 5G has channels up to 10 GHz
\end{itemize}
The increase in speed is partially due to better MA schemes, but mainly due to the bigger bandwidth.

Other options to increase capacity:
\begin{itemize}[noitemsep]
    \item Increase SNR
    \begin{itemize}[noitemsep]
        \item Increase power (limited by regulations)
        \item or reduce noise (choose a frequency with less background?)
    \end{itemize}
\end{itemize}

\section{Basic Antenna \& Propagation}
\textbf{Radio link design}
\begin{itemize}[noitemsep]
    \item What will be the power at the receiver (or SNR)?
    \item How far away can the antennae be?
    \item How much power is needed?
\end{itemize}
Answers culminate in a \textit{link budget} calculation

\textbf{Antennae}
\begin{itemize}[noitemsep]
    \item Circuitry generates the signal and amplifies to high power
    \item High power signal goes along a 'feed track'
    \begin{itemize}[noitemsep]
        \item Very little losses here
    \end{itemize}
    \item Then into the antenna
    \begin{itemize}[noitemsep]
        \item Can be directed, but not guided, so loses power quickly
        \item Not all power is radiated, some is lost
    \end{itemize}
\end{itemize}

\begin{flushright}[Lecture on 1.5]\end{flushright}

\textbf{Antenna Power}

Ignoring free space losses
\begin{itemize}[noitemsep]
    \item Assume total power remains constant as it propagates
    \item but spreads over a larger area (inverse square law)
    \item Consider \textit{power density} in $W/m^2$
\end{itemize}

With space losses
\begin{itemize}[noitemsep]
    \item Model with a 'propagation constant' $\gamma \ge 2$
    \item \begin{math}
    \frac{\text{Power density at distance }r_2}{\text{Power density at distance }r_1} = \frac{r_1}{r_2}^\gamma
    \end{math}
\end{itemize}

\textbf{Isotropic Reference Antenna}
\begin{itemize}[noitemsep]
    \item Assume a point source
    \item Radiates in all directions uniformly
    \item 100\% of input power is radiated
\end{itemize}
Not possible but useful for comparison purposes

\textbf{Omni-directional Antenna}
\begin{itemize}[noitemsep]
    \item Similar model but losses are allowable
    \item I npractice only possible in one plane
\end{itemize}

\textbf{Antenna Gain}
\begin{itemize}[noitemsep]
    \item Antennae are not amps - they don't actually have gain.
    \item However, a focused antenna delivers more power to the receiver than an isotropic tradiator, so there is a 'focusing gain'\\[1em]

    \begin{math}
    \text{Gain in a particular direction} := \frac{\text{Power density observed in that direction}}{\text{Power density expected from an isotropic radiator}}
    \end{math}\\[1em]

    \item Obviously normally measured in the direction of transmission
\end{itemize}

\subsection{Receiver Antennae}

\begin{itemize}[noitemsep]
    \item Collects electromagnetic waves
    \item May be directional - sensitive to waves from a certain direction
    \item Measure the aperture / collection area
    \begin{itemize}[noitemsep]
        \item Some antennae have an obvious physical aperture (eg. parabolic dish)
        \item Others have an 'effective aperture' $A$, such that
        \begin{math}
        P_{RX} = D_{RX}A
        \end{math}
        where $P_{RX}$ is the collected power and $D_{RX}$ is the power density of the incoming wave
        \end{itemize}
    \item Aperture efficiency
    \begin{itemize}[noitemsep]
        \item Even those with a physical aperture have an 'effective aperture', which is lower due to losses on the dish\\
        \begin{math}
            \text{Aperture efficiency} := \frac{\text{Effective aperture } A}{\text{Physical aperture}} < 1
        \end{math}
    \end{itemize}
    \item Always use 'effective aperture', which accounts for dish losses
\end{itemize}

\textbf{Reciprocity}

\begin{itemize}[noitemsep]
    \item Antennae can TX and RX
    \begin{itemize}[noitemsep]
        \item Same beam and shape each way
    \end{itemize}
    \item RX gain is equal to TX gain
    \item One expression for gain based on the aperture is\\
    \begin{math}
        G = \frac{4\pi A}{\lambda^2}
    \end{math}
    where $\lambda$ is wavelength.
\end{itemize}

\end{document}