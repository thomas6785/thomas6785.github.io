\documentclass[12pt]{article}

% Style guide
% Each 'section' is a chapter (PDF) on Brightspace
% They are 1-indexed in accordance with the notes on BS

\usepackage{enumitem}
\usepackage{amsmath}
\usepackage{parskip}
\usepackage{outlines}
\usepackage{framed}
\usepackage{graphicx}
\usepackage{xcolor}
\pagecolor{black}
\color{white}

\begin{document}

\tableofcontents

\begin{flushright}[Lecture on 1.2]\end{flushright}

\section{Introduction}
Dr. Declan Delaney

\subsection{Motivation and Introduction}
\begin{outline}[enumerate]
\1 Signals travel without wires
    \2 In this module, signals travel as radio waves\\
    (optical and acoustic systems ignored)
\1 Applications are mostly in communications
    \2 Signals modulated to carry information
    \2 Many familiar applications such as radar, navigation, etc.
\end{outline}

\begin{framed}
Example: Modern smart phone has approximately 9 distinct wireless systems. Try identifying them?


\begin{itemize}[noitemsep]
    \item NFC
    \item Cellulars
    \begin{itemize}[noitemsep]
        \item 2G
        \item 3G
        \item 4G
        \item 5G
    \end{itemize}
    \item GPS
    \item Bluetooth
    \item WiFi
    \item UWB
    \item Lidar
\end{itemize}
\end{framed}

\textbf{Advantages of Wireless}

\begin{itemize}[noitemsep]
    \item Mobility
    \item Good for one-to-many transmissions
    \item Cheap
\end{itemize}
Increasingly used for high capacity point-to-point links (cheaper than wired) (e.g. to serve remote areas)

\textbf{Advantages of Wired}

\begin{itemize}[noitemsep]
    \item Very little leakage
    \item No interference
    \item Multiple systems can operate adjacently without issue
\end{itemize}
but considerably more overheads.
Suitable for super high capacity lines (eg. fibre-optic transatlantic cables)

\subsection{The Wireless Spectrum}
The EM spectrum is a shared and limited resource.\\
Mostly regulated by government agencies.

Frequencies must be carefully given out, but can be reused at different locations as we will see.

Overview of a wireless system:
\begin{itemize}[noitemsep]
\item Start with raw data
\item Source coding (compression)
\item Channel coding (error detection \& error correction)
\item Modulation
\item TX
\item RX
\item Demodulation
\item Channel decoding
\item Source decoding
\end{itemize}

This module is mainly about modulation and TX/RX, the rest is information theory.

\subsection{Basics of Wireless Transmitter}
Amplifier to increase signal power enough to drive the antenna

\textbf{Multiple Access}
\begin{itemize}[noitemsep]
\item CSMA: Listen to the channel, send if it's clear
\item FDMA: Frequency divided MA
\item TDMA: Time divided MA
\end{itemize}

% TODO Look through remaining slides and pull some info re. labs, textbooks, etc.

\subsection{Assessment \& Delivery}
    \begin{tabular}{ l l l }
        \textbf{Component} & \textbf{Timing} & \textbf{Weight} \\
        Lab Assignments & Varied (4 labs) & 25\% \\
        Online BS quizzes & ? & 25\% \\
        Final Exam & & 50\%
    \end{tabular}

Open book final with emphasis on design and problem solving

\subsection{Module Outline}

\begin{itemize}[noitemsep]
    \item (1) Radio Link Design
    \begin{itemize}[noitemsep]
        \item Link budget?
        \item How far? How much power?
    \end{itemize}
    \item (2) Non-linear System
    \item (3) Frequency Generation and Synthesis
    \item (4) Transmitter Design
    \begin{itemize}[noitemsep]
        \item Requirements and specifications
        \item Transmitter architecture choices
    \end{itemize}
    \item (5) Noise
    \begin{itemize}[noitemsep]
        \item Sources of noise
        \item Noise analysis, low-noise design
    \end{itemize}
    \item (6) Receiver Design
    \begin{itemize}[noitemsep]
        \item Requirements and specifications
        \item Receiver architecture choices
    \end{itemize}
    \item (7) Transceiver Design
    \begin{itemize}[noitemsep]
        \item Transmitter and receiver combined!
    \end{itemize}
    \item (8) Antennas and Propagation
    \begin{itemize}[noitemsep]
        \item Review of antenna theory
        \item Practical antennas and propagation of radio waves
    \end{itemize}
    \item (9) System-Level Issues and Examples
\end{itemize}

\subsection{Textbooks}
Purely optional, module notes should be sufficient.

\begin{itemize}[noitemsep]
    \item "Microwave and RF Design of Wireless Systems"

    by David M. Pozar
    \item "Antennas"

    by John D. Kraus
\end{itemize}

\end{document}