\documentclass[12pt]{article}

% Style guide
% Each 'section' is a chapter (PDF) on Brightspace
% They are 1-indexed in accordance with the notes on BS

\usepackage{enumitem}
\usepackage{amsmath}
\usepackage{parskip}
\usepackage{outlines}
\usepackage{framed}
\usepackage{graphicx}
\usepackage{xcolor}
\pagecolor{black}
\color{white}

\begin{document}


\begin{flushright}
    [Lecture 1.2]
\end{flushright}

\section{Introduction}
Dr. Declan Delaney

\subsection{Motivation and Introduction}
\begin{outline}[enumerate]
\1 Signals travel without wires
    \2 In this module, signals travel as radio waves\\
    (optical and acoustic systems ignored)
\1 Applications are mostly in communications
    \2 Signals modulated to carry information
    \2 Many familiar applications such as radar, navigation, etc.
\end{outline}

\begin{framed}
Example: Modern smart phone has approximately 9 distinct wireless systems. Try identifying them?
\begin{outline}[enumerate]
    % TODO remove the gaps between these examples and make it more compact
\1 NFC
\1 Cellulars
    \2 2G
    \2 3G
    \2 4G
    \2 5G
\1 GPS
\1 Bluetooth
\1 WiFi
\1 UWB
\1 Lidar
\end{outline}
\end{framed}

\textbf{Advantages of Wireless}
\begin{itemize}[noitemsep]
\item Mobility
\item Good for one-to-many transmission
\item High-capacity point-to-point links (cheaper than wired) (e.g. to serve remote areas)
\end{itemize}

\textbf{Advantages of Wired}
\begin{itemize}[noitemsep]
\item Very little leakage
\item No interference
\item Multiple systems can operate adjacently without issue
\end{itemize}
but wired has much, much larger overheads.\\
Wired used for super high capacity lines (eg. fibre-optic transatlantic cables)

\subsection{The Wireless Spectrum}
The EM spectrum is a shared and limited resource.\\
Mostly regulated by government agencies.

Frequencies must be carefully given out, but can be reused at different locations as we will see.

Overview of a wireless system:
\begin{itemize} % TODO remove item separation
\item Start with raw data
\item Source coding (compression)
\item Channel coding (error detection \& error correction)
\item Modulation
\item TX
\item RX
\item Demodulation
\item Channel decoding
\item Source decoding
\end{itemize}

This module is mainly about modulation and TX/RX, the rest is information theory.

\subsection{Basics of Wireless Transmitter}
Amplifier to increase signal power enough to drive the antenna

\textbf{Multiple Access}
\begin{itemize}
\item CSMA: Listen to the channel, send if it's clear
\item FDMA: Frequency divided MA
\item TDMA: Time divided MA
\end{itemize}
3G uses FDMA and TDMA together.\\
3G uses CDMA\\
Also OFMA

% TODO Look through remaining slides and pull some info re. labs, textbooks, etc.

\subsection{Assessment \& Delivery}
    \begin{tabular}{ l l l }
        \textbf{Component} & \textbf{Timing} & \textbf{Weight} \\
        Lab Assignments & Varied (4 labs) & 25\% \\

    \end{tabular}
\subsection{Module Outline}
\subsection{Textbooks}

\end{document}