\documentclass[12pt]{article}

% Style guide

\setcounter{section}{-1}
\usepackage{lipsum}     % lol
\usepackage{float}      % allows [H] arguments for images to be placed where they are in the text
\usepackage{enumitem}   % For enumerated lists
\usepackage{amsmath}    % For math
\usepackage{parskip}    % Removes paragraph indent and adds paragraph gaps instead
\usepackage{framed}     % FRame box useful for examples and closing off sections
\usepackage{graphicx}   % For attaching images
\usepackage{xcolor}     % To set backgrounds and text colour to dark theme

% Set dark theme
\pagecolor{black}
\color{white}

\begin{document}

\begin{flushright}[Lecture on 1.1]\end{flushright}

\section{Introduction}
\textbf{Professor}: Nam Tran

\subsection{Topics}
\begin{itemize}[noitemsep]
    \item (0) Introduction
    \item (1) Overview of Random Variables
    \begin{itemize}[noitemsep]
        \item Random signals
    \end{itemize}
    \item (2) Filtering of Random Signals
    \begin{itemize}[noitemsep]
        \item Filtering is effectively convolution - how can we apply a convolution to a function without known exact output (i.e. a random signal)?
    \end{itemize}
    \item (3) Estimation Theory
    \item (4) Power Spectral Density Estimation
    \item (5) Wiener Filter Theory
    \item (6) Linear Estimator / Adaptive Linear Filter
    \item (7) Channel Equalisation
    \item (8) Image Processing
\end{itemize}
Lectures will not be recorded.

No material will be taught in week 12 (per plan).

\subsection{Assessment}
    \begin{tabular}{ |l l l l| }
        \hline
        \textbf{Component} & \textbf{Timing} & \textbf{Weight} & \textbf{Topic} \\ \hline
        Assignment 1       & Week 5        & 15\% & Topic (4) \\ \hline
        Assignment 2       & Week 8/9      & 15\% & Topic (6) \\ \hline
        Assignment 3       & Week 12       & 20\% & Topic (8) \\ \hline
        Final Exam         &               & 50\% & \\ \hline
    \end{tabular}


\section{Review of Random Variables}
\textbf{Random Variable}: Numerical description of the outcome of an experiment
\textbf{Sample Space}: All possible values of an RV
\textbf{Sample Point}: One possible value of an RV

If the sample space takes form $\{x_1,x_2,x_3,...,x_N\}$ we have a \textit{discrete} RV.

Capital letters are normally used.

\begin{framed}
\textbf{Example}
Sending a packet has 0.9 success rate.
Let X be the number of times you need to send a packet to get it through.

P(X=1) = 0.9
P(X=2) = 0.09
E(x) = 1.1111
\end{framed}

\begin{flushright}[Lecture on 1.2]\end{flushright}
Notes in a notebook currently. Need to decide whether to migrate them to LaTeX or migrate this to notebook. Probably the latter.

\end{document}