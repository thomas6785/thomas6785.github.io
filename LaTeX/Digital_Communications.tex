\documentclass[12pt]{article}

% Style guide

\usepackage{lipsum}     % lol
\usepackage{float}      % allows [H] arguments for images to be placed where they are in the text
\usepackage{enumitem}   % For enumerated lists
\usepackage{amsmath}    % For math
\usepackage{parskip}    % Removes paragraph indent and adds paragraph gaps instead
\usepackage{framed}     % FRame box useful for examples and closing off sections
\usepackage{graphicx}   % For attaching images
\usepackage{xcolor}     % To set backgrounds and text colour to dark theme

% Set dark theme
\pagecolor{black}
\color{white}

\begin{document}

\begin{flushright}[Lecture on 1.1]\end{flushright}

Prof. Mark Flanagan

\textbf{N.B.} Module content has changed since last year, previous exams may not be relevant

\subsection{Course Content}
Necessary background:
\begin{itemize}[noitemsep]
    \item Signals and Systems
    \begin{itemize}[noitemsep]
        \item Fourier, FFT, autocorrelation, PSD, etc.
    \end{itemize}
    \item Probabiliy Theory
    \begin{itemize}[noitemsep]
        \item Baye's Rule
    \end{itemize}
    \item Random Signals
    \begin{itemize}[noitemsep]
        \item AWGN, etc.
    \end{itemize}
    \item Linear algebra
    \begin{itemize}[noitemsep]
        \item Vector spaces, inner products, etc.
    \end{itemize}
\end{itemize}

Preliminary section: DMCs and the MAP rule.

Three mains sections:
\begin{itemize}[noitemsep]
    \item Signal Space Analysis
    \item Modulation Techniques
    \item Wireless Communications
\end{itemize}

Recommended textbook: "Communication Systems" by Simon Haykin (4th ed)

\subsection{Assessment}
    \begin{tabular}{ |l l| }
        \hline
        \textbf{Component} & \textbf{Weight} \\ \hline
        MATLAB Assignment 1   & 10\% \\ \hline
        MATLAB Assignment 2   & 10\% \\ \hline
        Final Exam            & 80\% \\ \hline
    \end{tabular}

\section{Discrete Memoryless Channel}
Idealised channel with no memory.

Symbols $x_j$ are transmitted, symbols $y_j$ are received for $0<j<M$.

\begin{itemize}[noitemsep]
    \item Probability map of TX symbols to RX symbols
    \item 'a priori' probability of each symbol $x_j$ being transmitted
    \item Use Bayes' and cleverness to get 'a posteriori' probability of $x_j$ for a given $y_j$
\end{itemize}
\textbf{Question}: Suppose $y_k$ is observed at the output. What is the \textit{optimum decision rule}?

\textbf{Answer}: We define the optimum decision rule as follows: The receiver sets its decision $\hat{x}$ to be the \textit{most likely transmitted} symbol. This is called the 'maximum a posteriori' (MAP) decision rule. Mathematically:
\[\hat{x}=\text{argmax}_{x_j} P(x_j|y_k) \]

Using Bayes' rule,
\[P(x_j|y_k) = \frac{P(y_k|x_j) \times P(x_j) }{P(y_k)}\]

The first factor is given by the probability map of symbol errors. The second is given by the 'a priori' probabilities.

\end{document}