\documentclass[12pt]{article}

% Style guide

\usepackage{lipsum}     % lol
\usepackage{float}      % allows [H] arguments for images to be placed where they are in the text
\usepackage{enumitem}   % For enumerated lists
\usepackage{amsmath}    % For math
\usepackage{parskip}    % Removes paragraph indent and adds paragraph gaps instead
\usepackage{framed}     % FRame box useful for examples and closing off sections
\usepackage{graphicx}   % For attaching images
\usepackage{xcolor}     % To set backgrounds and text colour to dark theme

% Set dark theme
\pagecolor{black}
\color{white}

\begin{document}

\begin{flushright}[Lecture on ?.?]\end{flushright}

\section{Introduction}
\textbf{Professor}: ???

\subsection{Motivation and Introduction}
\begin{itemize}[noitemsep]
    \item Insert module information here.
    \item You can use the enumerate environment, but it's probably easier to number them by hand for more flexibility anyway.
\end{itemize}

\subsection{Assessment \& Delivery}
    \begin{tabular}{ |l l l| }
        \hline
        \textbf{Component} & \textbf{Timing} & \textbf{Weight} \\ \hline
        Item One Here       & Week 1        & 25\% \\ \hline
        Item Two Here       & Week 2        & 30\% \\ \hline
        Final Exam          &               & 45\% \\ \hline
    \end{tabular}


\subsection{Module Outline}

\subsection{Textbooks}

\subsection{Sample Notes}
Notes may look something like this:
\begin{framed}
\textbf{This bit of text appears in a frame.}\\
This could be useful to clearly demarcate the beginning or end of a derivation or sample problem.

\[ x=1 \]

You can put math in too!

\end{framed}

Normally we put slide titles or concepts like this.
\begin{itemize}[noitemsep]
    \item Press Ctrl+L to create a bulleted list
    \item Like that
    \begin{itemize}[noitemsep]
        \item You can press it within a list for sub-points
        \item Like that
        \begin{itemize}[noitemsep]
            \item or even sub-sub points!
        \end{itemize}
    \end{itemize}
    \item That's how that works.
\end{itemize}

Press Ctrl+Shift+V then ENTER to paste images.
\begin{figure}[H]
    \centering
    \includegraphics[width=0.5\linewidth]{images/image3.png}
    %\caption{}
    %\label{fig:}
\end{figure}

The [H] argument ensures the image appears in the right place.

Use Ctrl+B or Ctrl+I to insert \textbf{bold} and \textit{italics}.

\begin{flushright}[Lecture on 3.4]\end{flushright}
Lastly, try to include lecture dates each time you start editing to make it easier to revise.

To do this, enter 'lec' then hit tab to get the snippet.

\end{document}